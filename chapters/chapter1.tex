\chapter{Tehnologii}

În cadrul acestui proiect s-au folosit urmatoarele tehnologii :
\begin{itemize}
\item \textbf{Unity 2019.1b} : motorul grafic folosit pentru construirea și randarea jocului.
\item \textbf{Blender} : program \textit{open source} folosit pentru modelarea 3D si contrucția obiectelor ce au fost ulterior importate în motorul grafic.
\item \textbf{Math.NET} : librarie \textit{open source} folosită pentru realizarea algoritmului de învățare automată deoarece acesta se bazează foarte mult pe lucru cu matrici.
\item \textbf{C\#} : limbajul nativ folosit de Unity pe lângă JavaScript , decizia fiind luată pur din motive de viteză.
\end{itemize}

Din multitudinea de motoare grafice disponibile am ales \textbf{Unity} doarece e o tehnologie cu care am mai lucrat , fiind conștient de capabilitățile acestuia. Multitudinea de unelte îl fac foarte ușor de recomandat și împreuna cu magazinul de \textit{assets} m-au ajutat foarte mult în procesul de construcție al acestui proiect. De asemenea \textbf{Unity} dispune de o documentație foarte bogată si bine pusă la punct , calități cheie pentru aprofundarea acestui motor grafic.\par

Câteva exemple de unelte \textit{in-house} de care dispune motorul grafic ar fi , motorul de simularea a fizicii , motorul audio , sistemul de prefabricate , sistemul de iluminare și randare etc. Fara folosirea unei tehnologii ca \textbf{Unity} , acest proiect nu ar fi fost posibil într-un timp atăt de scurt , timpul de dezvoltare putând fi chiar triplat. Singurul dezavantaj imediat al acestui motor grafic este natura sa \textit{closed source}.\par

Pentru modelele 3D folosite în cadrul jocului a fost necesară folosirea unei editor și anume \textbf{Blender}. Deși mare parte din obiectele importate în joc sunt preluate de pe magazinul integrat în \textbf{Unity} , există anumite obiecte ce au fost modelate de către mine. Cu acest prilej am aprofundat anumite tehnici ce implică modelarea 3D , unul dintre cele mai importante find \textit {UV unwrapping}.\par

Am ales \textbf{Blender} datorită politcii \textit{open source} și a documentației vaste disponibile, fiind foarte ușor de lucrat în acesta, neputând aduce aplicația la un nivel dorit de finisaj fară acestă unealtă.\par

Din punct de vedere al librăriilor externe am folosit \textbf{Math.NET} , importată în \textbf{Unity} folosind \textbf{NuGet}. Necesitatea acestei librări este dator algoritmului de invățare automată ce se bazează foarte mult pe lucrul și manipularea matricilor, de la adunarea cu un scalar până la înmulțirea/împărțirea element cu element. Detaliile legate de acest algoritm împreuna cu avantajele și dezavantajele acestuia vor fi discutate în secțiunea teoretică.\par

Datorită motorului grafic folosit și a limitărilor acestuia întregul proiect a fost construit folosind \textbf{C\#} , alternativa sa fiind \textbf{JavaScript}. Din cauza experienței cu \textbf{C\#} și a performanței ridicate am luat decizia de a folosi acest limbaj.\par


\section{Structura proiectului}

Aplicația creată este împărțită în doua etape , generare și instanțiere. Din cauza unor motive de performanță fiecare zonă  , deșertică , rurală și urbană are propriul său generator. De asemenea fiecare tip de platforma foloșeste propriul său model Markov însumând astlfel un total de șase generatori , ce se ocupă doar cu producerea unei secvențe specifice de obiecte. Pentru partea de instanțiere am fost nevoit sa construiesc un sistem ce facilitează instațiera obiectelor generate , plasăndule pe platformă în sensul acelor de ceasornic.\par

Aplicația fiind facută în \textbf{Unity} fișierele și directoarele au o structură arborescentă după cum urmează: \par

\dirtree{%
.1 \textbf{Assets}.
.2 \textbf{Animations}.
.2 \textbf{Generators}.
.2 \textbf{HMM}.
.2 \textbf{Sound}.
.2 \textbf{Scripts}.
.2 \textbf{Textures}.
.2 \textbf{VolumetricLights}.
.1 \textbf{Packages}.
.2 \textbf{PostProcessing}.
}
\par

Directorul de \textbf{Assets} conține toate elementele necesare aplicației , de la librăria folosită pentru generea obiectelor până la modulele de sunet și interfață iar directorul \textbf{Packages} conține cele mai importante uneltele folosite pentru realizarea si finisajul jocului importate direct de către \textbf{Unity}.\par

\textbf{Animations}  este directorul responsabil cu toate animațiile necesare jocului , meniu , tranziți și altele.\par

\textbf{Generators} aici se află cei sașe generatori folosiți pentru construirea mediului.\par

\textbf{HMM} este directorul ce conține scripturile necesare pentru construirea modelului Markov și a antrenării lui.\par

\textbf{Sound} conține toate efectele de sunet și muzica din joc.\par

\textbf{Scripts} aici sunt stocate toate fișierele sursă ce modeleaza comportamentul jucatorului , al meniurilor și a sistemelui de generare/instanțiere , sumându-se în total la optsprezece fișiere.\par

\textbf{Textures} directorul ce conține toate fisiere de tip \textit{Albedomap} , \textit{Heightmap} și \textit{Occlusionmap} folosite în cadrul materialelor ce au fost ulterior importate pe obiecte.\par

\textbf{VolumetricLights} directorul ce contine fișierele preluate de pe github pentru lumina volumetrică.\par

\textbf{PostProcessing} modulul folosit de \textit{Unity} pentru a putea integra efecte precum corectare de culoare , \textit{bloom} , \textit{motion blur} și multe altele.\par


\section{Titlul secțiunii 2}

Pellentesque pulvinar pellentesque habitant morbi tristique senectus et. Ornare suspendisse sed nisi lacus sed viverra tellus in hac. Non sodales neque sodales ut etiam sit. In hendrerit gravida rutrum quisque non. Diam quam nulla porttitor massa id neque aliquam. Diam sit amet nisl suscipit adipiscing bibendum est ultricies integer. Cras fermentum odio eu feugiat pretium nibh ipsum. Egestas integer eget aliquet nibh praesent tristique magna. Porttitor eget dolor morbi non arcu risus quis varius quam. Gravida rutrum quisque non tellus orci. Diam volutpat commodo sed egestas egestas.