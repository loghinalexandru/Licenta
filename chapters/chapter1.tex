\chapter{Tehnologii}

În cadrul acestui proiect s-au folosit urmatoarele tehnologii :
\begin{itemize}
\item \textbf{Unity 2019.1b} : motorul grafic folosit pentru construirea și randarea jocului.
\item \textbf{Blender} : program \textit{open source} folosit pentru modelarea 3D si contrucția obiectelor ce au fost ulterior importate în motorul grafic.
\item \textbf{Math.NET} : librarie \textit{open source} folosită pentru realizarea algoritmului de învățare automată deoarece acesta se bazează foarte mult pe lucru cu matrici.
\item \textbf{C\#} : limbajul nativ folosit de Unity pe lângă JavaScript , decizia fiind luată pur din motive de viteză.
\end{itemize}

Din multitudinea de motoare grafice disponibile am ales \textbf{Unity} doarece e o tehnologie cu care am mai lucrat , fiind conștient de capabilitățile acestuia. Multitudinea de unelte îl fac foarte ușor de recomandat și împreuna cu magazinul de \textit{assets} m-au ajutat foarte mult în procesul de construcție al acestui proiect. De asemenea \textbf{Unity} dispune de o documentație foarte bogată si bine pusă la punct , calități cheie pentru aprofundarea acestui motor grafic.\par

Câteva exemple de unelte \textit{in-house} de care dispune motorul grafic ar fi , motorul de simularea a fizicii , motorul audio , sistemul de prefabricate , sistemul de iluminare și randare etc. Fara folosirea unei tehnologii ca \textbf{Unity} , acest proiect nu ar fi fost posibil într-un timp atăt de scurt , timpul de dezvoltare putând fi chiar triplat. Singurul dezavantaj imediat al acestui motor grafic este natura sa \textit{closed source}.\par

Pentru modelele 3D folosite în cadrul jocului a fost necesară folosirea unei editor și anume \textbf{Blender}. Deși mare parte din obiectele importate în joc sunt preluate de pe magazinul integrat în \textbf{Unity} , există anumite obiecte ce au fost modelate de către mine. Cu acest prilej am aprofundat anumite tehnici ce implică modelarea 3D , unul dintre cele mai importante find \textit {UV unwrapping}.\par

Am ales \textbf{Blender} datorită politcii \textit{open source} și a documentației vaste disponibile, fiind foarte ușor de lucrat în acesta, neputând aduce aplicația la un nivel dorit de finisaj fară acestă unealta.\par

Din punct de vedere al librăriilor externe am folosit \textbf{Math.NET} , importată în \textbf{Unity} folosind \textbf{NuGet}. Necesitatea acestei librări este dator algoritmului de invățare automată ce se bazează foarte mult pe lucrul și manipularea matricilor, de la adunarea cu un scalar până la înmulțirea/împărțirea element cu element. Detaliile legate de acest algoritm împreuna cu avantajele și dezavantajele acestuia vor fi discutate în secțiunea teoretică.\par


\section{Titlul secțiunii 1}

Id donec ultrices tincidunt arcu non sodales neque. Integer eget aliquet nibh praesent. Euismod in pellentesque massa placerat duis ultricies lacus sed. Mauris ultrices eros in cursus turpis massa. Integer quis auctor elit sed vulputate mi. Nibh ipsum consequat nisl vel pretium lectus quam id leo. Vel elit scelerisque mauris pellentesque pulvinar pellentesque. Suscipit tellus mauris a diam maecenas. Ultrices eros in cursus turpis massa tincidunt. Tristique senectus et netus et malesuada fames ac turpis egestas. Suspendisse interdum consectetur libero id faucibus nisl tincidunt eget. Sed risus pretium quam vulputate dignissim suspendisse in. Donec adipiscing tristique risus nec feugiat in fermentum posuere. A lacus vestibulum sed arcu non odio euismod lacinia at.

\section{Titlul secțiunii 2}

Pellentesque pulvinar pellentesque habitant morbi tristique senectus et. Ornare suspendisse sed nisi lacus sed viverra tellus in hac. Non sodales neque sodales ut etiam sit. In hendrerit gravida rutrum quisque non. Diam quam nulla porttitor massa id neque aliquam. Diam sit amet nisl suscipit adipiscing bibendum est ultricies integer. Cras fermentum odio eu feugiat pretium nibh ipsum. Egestas integer eget aliquet nibh praesent tristique magna. Porttitor eget dolor morbi non arcu risus quis varius quam. Gravida rutrum quisque non tellus orci. Diam volutpat commodo sed egestas egestas.