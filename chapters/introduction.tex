\chapter*{Introducere} 
\addcontentsline{toc}{chapter}{Introducere}

Având în vedere necesitatea creării unor medii căt mai versatile ce respectă anumite condiții sau restricții fizice s-au dezvoltat anumite tehnici pentru a facilita construcția automată și randomizată a oricărui element ce intră în componența unui joc , de la modele tridimensionale până la coloana sonoră.\par

Tema lucrării s-a ivit din cauza necesității de adaptare rapidă a condițiilor folosite pentru generarea obiectelor din componența unui joc. Unul din avantajele acestei abordări este flexibilitatea oferită de modelul Markov fiind capabil să adapteze constrăngerile folosite la generare cu o simplă reantrenare a modelului.\par

Aplicația prezentă este construită folosind un motor grafic peste care am dezvoltat o librărie capabilă să reprezinte un model Markov și să îl antreneze. Folosindu-mă de această librărie , pot genera o secvență de obiecte ce încearcă să respecte fidel constrângerile folosite la antrenare.\par

Procesul de generare procedurală este compus din două etape. Generarea obiectelor și plasarea lor convenabil în șpațiul virtual. Am încercat în special să decuplez cele două etape pentru o mai bună modularizare și o coeziune ridicată. Cele două sisteme lucrează independent unul față de celălalt , relația dintre cele două find una de agregare.

Cele două sisteme sunt puse în funcțiune pentru a genera în timp real mediul pentru jucător, mediu ce este împărțit în trei zone de interes , urbana , rurală și deșertică. De asemenea pentru a adauga complexitate se alternează între trei tipuri de platforme ce simulează un drum drept, un drum cu viraj la stănga sau un drum cu viraj la dreapta.\par