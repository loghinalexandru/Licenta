\chapter*{Concluzii} 
\addcontentsline{toc}{chapter}{Concluzii}


Scopul principial al acestei aplicații a fost de a incerca și valida o nouă abordare pentru generare procedurală și de a crea o librărie pentru acest lucru. Din fericire aplicația demonstrează aptitudinile și flexibilitatea acestei abordări împreună cu ușurința utilizării acesteia. Proiectul este $open-source$ find posibilă uilizarea lui în cadrul oricărei aplicații.\par

Odată cu terminarea acestei lucrări ce a venit chiar din necesitate, cu siguranță voi utiliza librăria în următoarele proiecte ce utilzează $Unity$, fiind o metodă foarte ușoară de a antrena și genera o secvență de obiecte.\par

Bineînteles există multe optimizări ce s-ar putea face în cadrul librăriei, primul lucru fiind implementarea metodei lui $Alias$ pentru eșantionarea dintr-o distribuție probabilistă discretă, reducând timpul necesar pentru tranziții și emisii dar acest lucru ar putea fi luat în considerare doar dacă versiunea curentă ce folosește o căutare binară impune probleme de performanță, pentru aplicația descrisă în această teză fiind suficientă varianta din urmă.\par
