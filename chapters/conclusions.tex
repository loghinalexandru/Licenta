\chapter*{Concluzii} 
\addcontentsline{toc}{chapter}{Concluzii}

Scopul principial al acestei aplicații a fost de a incerca și valida o nouă abordare pentru generare procedurală și de a crea o librărie pentru acest lucru. Din fericire aplicația demonstrează aptitudinile și flexibilitatea acestei abordări împreună cu ușurința de utilizare. Proiectul este $open \ source$ find posibilă uilizarea lui în cadrul oricărei aplicații.\par

Odată cu terminarea acestei lucrări ce a venit din necesitate, cu siguranță voi utiliza librăria în următoarele proiecte ce utilzează \textbf{Unity}, fiind o metodă foarte ușoară de a genera o secvență de obiecte.\par

Bineînțeles există multe optimizări ce s-ar putea face în cadrul librăriei, primul lucru fiind implementarea \textbf{metodei lui Alias} pentru eșantionarea dintr-o distribuție probabilistă discretă, reducând timpul necesar pentru tranziții și emisii.\par 

Acest lucru ar putea fi luat în considerare doar dacă versiunea curentă ce folosește o căutare binară impune probleme de performanță, pentru aplicația descrisă în această teză a fost suficientă varianta din urmă.\par

De asemena cum am mai menționat în cadrul lucrării, agentul inteligent folosit de adversar este unul foarte primitiv, acesta ar putea fi îmbunătățit substanțial prin diverse implementări mai mult sau mai puțin complexe. Nu am pus accentul pe acest aspect doarece motivația principală era generarea și instanțierea folosind modelul Markov cu stări invizibile.\par

Legat de platformă pentru care este destinată aplicația, din motive de complexitate al tenhnicilor grafice aceasta este diponibilă doar pe desktop, neavând în plan celelalte platforme datorită puterii de procesare reduse. Se poate încerca portare pe alte platforme, dar ar trebui diminuat major din tehnicile de consolidare a graficii ce sunt folosite în cadrul platformei actuale.\par

Cu acest proiect am încercat să demonstrez abilitățile modelului și să încurajez implicarea mai multor developeri în dezvoltarea acestei tehnici pentru generare.\par
