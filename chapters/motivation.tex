\chapter*{Motivație} 
\addcontentsline{toc}{chapter}{Motivație}

Datorită pasiunii mele pentru \textit{gamedev} și algoritmică am decis să realizez o modalitate de a genera procedural un mediu cât mai diversificat, ce poate fi modificat în timp real. Folosirea modelelor Markov a venit din necesitatea de a scăpa cumva de metoda clasică prin care se generează un șir de obiecte într-un mod aleator iar apoi se încearcă validarea acestuia după anumite constrângeri. Folosind acest model stocastic putem controla apariția unei subsecvențe prin stabilirea unei anumite probabilități de a se produce așadar eliminând cu totul pasul de validare.\par

Am decis așadar să încerc o abordare nouă , folosindu-mă de un model Markov ce a fost antrenat să respecte anumite reguli pentru a genera spațiul 3D. Scopul acestei lucrări este de a aduce o nouă perspectivă asupra generării procedurale și de a demonstra validitatea acestei abordări.\par

Pentru a arăta capabilitățile modelului, în cadrul proiectului a fost dezvoltată o librărie ce ușurează procesul de generare. De asemenea a fost nevoie și de un sistem ce facilitează plasarea obiectelor în mediul 3D, modelul Markov fiind folosit doar ca și generator. Odată funcționale mi-am îndreptat atenția asupra interfeței grafice și a interacțiunii dintre jucător și joc.\par

Datorită motorului grafic folosit și a instrumentelor de care dispune, timpul necesar pentru crearea jocului a fost îmbunătățit substanțial. De asemena cu prilejul acestei aplicații am avut posibilitatea de a aprofunda și testa diferite tehnici de redare și postprocesare. Câteva dintre aceste elemente includ SSR \footnote{Screen Space Reflection}, lumină volumetrică, reflexii în timp real, corectare de culoare și multe altele.\par

Din punct de vedere al generării procedurale folosind modele Markov există articole ce prezintă folosirea acestora însă nu am găsit nici o librărie pentru un motor de joc ce implementează această abordare într-un mod facil și eficient. Ba chiar mai mult de obicei orice încercare descrisă în aceste publicații omit complet partea de învățare automată a acestor modele, ceea ce m-a determinat să abordez această temă.\par

În mod normal generarea procedurală folosește în spate un algoritm precum $Diamond-square$, sau un generator de zgomot precum $Perlin \ noise$. Abordarea folosind modelele Markov este mai facilă doarece pot fi foarte ușor antrenate să respecte un anumit set de constrângeri în mod dinamic, permițând o mai mare expresivitate și putere de modelare. \par

Desigur ca și celelalte metode acesta prezintă anumite dezavantaje, deși minore, pot influența decizia de a folosi acest tip de generare. Librăria creată în cadrul acestui proiect are ca scop promovarea acestui model căt și validarea lui, prin ușurința de utilizare căt și flexibilitatea oferită.\par

Unul din aceste dezavantaje este dezvoltare sa în \textbf{Unity} doarece are anumite dependențe de structura acestuia, dar ea poate fi ușor portată pe altă platformă întrucât în cadrul secțiunii teoretice este prezentat în detaliu un pseudocod ce descrie modul de funcționare. Pe lângă modelul Markov, algoritmul ce se ocupă cu antrenare acestora este foarte ușor de înțeles și folosit, necesitând doar o librărie de calcul numeric ce lucrează cu matrici.\par  

Un obiectiv în crearea acestui pachet ce modelează un model Markov cu stări invizibile a fost eficiența, tocmai pentru a putea fi folosit într-o gamă mai largă de aplicații, incluzând cele de tip mobile. Cu toate acestea aplicația este solicitantă grafic din cauza tehnicilor de îmbunătățire a imaginii, așadar un port pentru dispozitivele portabile este exclus, fiind posibil doar dacă se reduce din complexitatea vizuală.\par

Odată cu finalizarea acestei aplicații, din cauza naturii sale \textit{open source}, se va putea folosi librăria creată în cadrul acesteia pentru orice tip de proiect fără constrăngeri, fiind ușor adaptabilă pentru orice sarcină ce necesită generare procedurală, de la muzică până la construcția unui mediu virtual, librăria find ușor de folosit și axată pe eficiență.\par
