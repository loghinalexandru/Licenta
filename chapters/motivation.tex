\chapter*{Motivație} 
\addcontentsline{toc}{chapter}{Motivație}

Datorită pasiunii mele pentru \textit{gamedev} și algoritmică am decis să realizez o modalitate de a genera procedural un mediu căt mai diversificat. Folosirea modelelor Markov a venit din necesitatea de a scăpa cumva de metoda clasică prin care se generează un șir de obiecte într-un mod aleator iar apoi se încearcă validarea acesteia după anumite constrângeri. Folosind acest model stocastic putem controla apariția unei anumite subsecvențe prin stabilirea unei anumite probabilități de a se produce așadar eliminând cu totul pasul de validare.\par

Scopul principal al proiectului a fost dezvoltarea unei librări ce ușurează procesul de generare. De asemenea a fost nevoie și de un sistem ce facilitează plasarea obiectelor în mediul 3D , modelul Markov fiind folosit doar ca și generator. Odată  funcționale mi-am îndreptat atenția asupra interfeței grafice și a interacțiunii dintre jucător și joc.\par

Datorită motorului grafic folosit și a instrumentelor de care dispune, timpul necesar pentru crearea jocului a fost îmbunătățit substanțial. De asemena cu prilejul acestei aplicații am avut posibilitatea de a aprofunda și testa diferite tehnici de randare și post-procesare. Câteva dintre aceste elemente includ SSR \footnote{Screen Space Reflection}, lumină volumetrică, reflexii în timp real, corectare de culoare și multe altele.\par

Odată cu finalizarea acestei aplicații , din cauza naturii sale \textit{open source} , se va putea folosi librăria creată în cadrul acesteia pentru orice tip de proiect fără constrăngeri, fiind ușor adaptabilă pentru orice sarcină ce necesită generare procedurală, de la muzică până la construcția unui mediu virtual, librăria find ușor de folosit și axata pe eficiență.\par

BAGĂ CE ADUCE NOU APLICAȚIA ȘI CE AM INCERCAT!!!

RESCALEAZĂ BUCĂȚILE DE COD ȘI IMAGINI