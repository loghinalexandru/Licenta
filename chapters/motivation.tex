\chapter*{Motivație} 
\addcontentsline{toc}{chapter}{Motivație}

Datorita pasiunii mele pentru \textit{gamedev} și algoritmică am decis să realizez o modalitatea de a genera procedural un mediu căt mai diversificat. Folosirea modelelor Markov a venit din necesitatea de a scăpa cumva de metoda clasică prin care se genereaza un șir de obiecte într-un mod aleator iar apoi se încearca validarea acesteia după anumite constrăngeri. Folosind acest model stocastic putem controla apariția unei anumite subsecvențe prin stabilirea unei anumite probabilități de a se produce așadar eliminănd cu totul pasul de validare.\par

Scopul principal al proiectului a fost dezvoltarea unei librări ce ușurează procesul de generare. De asemenea a fost nevoie și de un sistem ce facilitează plasarea obiectelor în mediul 3D , modelul Markov fiind folosit doar ca și generator. Odată funcționale mi-am îndreptat atenția asupra interfeței grafice și a interacțiunii dintre jucător și joc.\par

Datorită motorului grafic folosit și a instrumentelor de care dispune , timpul necesar pentru crearea jocului a fost îmbunătățit substanțial. De asemena cu prilejul acestei aplicații am avut posibilitatea de a aprofunda și testa diferite tehnici de randare și postprocesare. Câteva dintre aceste elemente includ SSR \footnote{Screen Space Reflection} , lumina volumetrică  , reflexii în timp real , corectare de culoare și multe altele.\par

Odată cu finalizarea acestei aplicații , din cauza naturii sale \textit{open-source} , se va putea folosi librăria creata în cadrul acesteia pentru orice tip de proiect fără constrăngeri , fiind ușor adaptabilă pentru orice sarcină ce necesită generare procedurala , de la muzică până la construcția unui mediu virtual, librăria find ușor de folosit și axata pe eficiență.\par